% Place abstract below.
 
Bilinguals co-activate lexical and syntactic alternatives in both languages when reading or speaking in one language. This series of experiments on L1 Spanish -- L2 English bilinguals shows that syntactic information that is specific to Spanish modulates lexical co-activation. A set of word reading studies shows that bilinguals experience cross-language co-activation when reading language ambiguous words (i.e., cognates and homographs) in isolation and in sentence context. However, the magnitude of cross-language co-activation in sentence context depended on the type of sentence structure in which the target words were embedded. Active and passive structures share word order in English and Spanish and exhibit cross-language syntactic priming (CLSP; a measure of syntactic co-activation) between the two languages, suggesting that they are syntactic structures that are non-specific to Spanish and English. In contrast, prepositional object structures optionally differ in word order across English and Spanish and could therefore be considered language-specific (but the propensity for CLSP has not been tested in previous empirical studies). Cross-language effects shifted when comparing the language non-specific actives and passives to the language-specific datives and when comparing the dative condition with non-overlapping word order to the dative condition with overlapping word order, indicating that that the syntactic information provided by a sentence context influences word recognition and cross-language co-activation.

A follow-up CLSP experiment provides independent evidence that actives and passives are language non-specific structures and that datives are language-specific. Participants were more likely to produce a passive structure in Spanish after they heard a passive structure in English. This cross-language syntactic priming effect suggests that bilinguals co-activate these structures and that the structures share representation across the two languages. In contrast, prepositional object dative structures showed no evidence of cross-language priming, indicating that the two structures are not shared and do not become co-activated during processing.  

Overall, the results of this dissertation show that bilinguals share representations and experience cross-language co-activate at both the lexical level and the syntactic level. Further, the results indicate that there is considerable interaction between the two levels. Word order appears to function as a cue that allows a bilingual speaker to differentiate between the two languages, reducing co-activation at each level. The results here also highlight the complicated nature of the interactions between language co-activation and other factors including sentence context, executive function ability, and language proficiency. 
